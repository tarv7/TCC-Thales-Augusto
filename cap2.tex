\chapter{Referencial Teórico}
Lorem ipsum dolor sit amet, consectetur adipiscing elit. Vivamus eu magna cursus, mattis velit et\cite{iso_cut_test_1114}, pulvinar lorem. Integer ut nulla eget tellus luctus pellentesque. Nullam fermentum arcu sed tristique congue. Sed ut augue a turpis imperdiet maximus vitae accumsan nisi. Fusce vitae dapibus orci. Pellentesque rutrum tincidunt turpis, id blandit libero lacinia eu. Nunc imperdiet dolor scelerisque ex tristique, nec elementum sem iaculis. Suspendisse eu leo sapien. Vivamus facilisis ultrices sollicitudin. Nunc suscipit, velit eget pretium rhoncus\cite{cocoa_quality_requeriments_2016}, erat nisl facilisis nulla, quis ultricies orci tellus sed ante. Sed pellentesque, nisl ac venenatis feugiat, augue enim consequat augue, sed porttitor sem purus in enim. Praesent vel tellus eu libero vehicula sodales. Integer tincidunt faucibus sollicitudin. Sed dapibus convallis metus, vel accumsan sem convallis ut.


\section{Conteúdo aqui, conteúdo aqui}
Lorem ipsum dolor sit amet, consectetur adipiscing elit. Vivamus eu magna cursus, mattis velit et, pulvinar lorem. Integer ut nulla eget tellus luctus pellentesque. Nullam fermentum arcu sed tristique congue. Sed ut augue a turpis imperdiet maximus vitae accumsan nisi. Fusce vitae dapibus orci. Pellentesque rutrum tincidunt turpis, id blandit libero lacinia eu. Nunc imperdiet dolor scelerisque ex tristique, nec elementum sem iaculis. Suspendisse eu leo sapien. Vivamus facilisis ultrices sollicitudin. Nunc suscipit, velit eget pretium rhoncus, erat nisl facilisis nulla, quis ultricies orci tellus sed ante. Sed pellentesque, nisl ac venenatis feugiat, augue enim consequat augue, sed porttitor sem purus in enim. Praesent vel tellus eu libero vehicula sodales. Integer tincidunt faucibus sollicitudin. Sed dapibus convallis metus, vel accumsan sem convallis ut.

\section{Trabalhos correlatos}
\label{sec:trabalhos_correlatos}

Lorem ipsum dolor sit amet, consectetur adipiscing elit. Vivamus eu magna cursus, mattis velit et\cite{silva2013caracterizaccao}, pulvinar lorem. Integer ut nulla eget tellus luctus pellentesque. Nullam fermentum arcu sed tristique congue. Sed ut augue a turpis imperdiet maximus vitae accumsan nisi. Fusce vitae dapibus orci. Pellentesque rutrum tincidunt turpis, id blandit libero lacinia eu. Nunc imperdiet dolor scelerisque ex tristique, nec elementum sem iaculis. Suspendisse eu leo sapien. Vivamus facilisis ultrices sollicitudin. Nunc suscipit, velit eget pretium rhoncus, erat nisl facilisis nulla, quis ultricies orci tellus sed ante. Sed pellentesque, nisl ac venenatis feugiat, augue enim consequat augue, sed porttitor sem purus in enim. Praesent vel tellus eu libero vehicula sodales. Integer tincidunt faucibus sollicitudin. Sed dapibus convallis metus, vel accumsan sem convallis ut.

%\subsection{Guia do beneficiamento do cacau de qualidade}
%O guia \cite{guia_beneficiamento_cacau_2013}, fala da importância do beneficiamento das sementes de cacau e o passo a passo, com dicas e técnicas, para que o cacau seja produzido com elevado padrão de qualidade. 

%Em uma das seções, fala-se sobre como realizar a prova de corte, indicando como deve ser a escolha das amêndoas, a quantidade de porções e o peso de cada porção para que a seleção seja da forma mais aleatória possível, e então fala sobre a disposição das amêndoas para a análise visual, além de recomendar a análise olfativa e do paladar. 

%Possui algumas imagens indicando o passo a passo, as imagens de um exemplar de amêndoa de cada classe, e disponibiliza uma tabela de avaliação de qualidade, onde o produtor insere informações como a data da colheita, aroma, umidade e a classifica visualmente.

%O guia possui dados que auxiliam também a identificar a classe dado a tolerância máxima de percentuais de defeitos para amêndoas de cacau comercial, segundo a Instrução Normativa nº 38/2008 (MAPA).

%Guia do beneficiamento

%\subsubsection{\textit{Cocoa Beans Industry Quality Requirements}}
%A publicação \cite{cocoa_quality_requeriments_2016} é subdividido em três partes, onde a primeira fala sobre os aspectos de qualidade das sementes de cacau, a segunda que fala sobre os padrões de qualidade de cacau internacionais e outros padrões, e a terceira parte que aborda sobre como aspectos da produção do cacau afetam os requerimentos de qualidade.

%Na seção da prova de corte, dita as regras que devem ser seguidas para que haja a comprovação da qualidade das amêndoas, que são citadas em tópicos anteriores. É dito sobre o que é a prova de corte, qual a finalidade, a quantidade de amostras que deve ser selecionada, e sobre o corte a ser feito em cada amêndoa. Sobre as categorias classificadas pela ISO, e sobre o uso de apenas uma parte delas.

%Na mesma seção fala sobre outra forma de realizar o teste de qualidade, que é a contagem de sementes, onde é verificado a quantidade de sementes multiplicado por 100, e esse produto é dividido pela massa total dessas sementes selecionadas.

%
%\subsection{Cartilha de classificação CEPLAC}
%A cartilha \cite{cartilha_ceplac} informa sobre os passos do correto beneficiamento do cacau, desde sua colheita, a disposição em bananeiras, o descanso para consentração dos açucares, a quebra do cacau, o transporte para a casa de fermentação, o processo de fermentação e a secagem e são exibidos os defeitos das amêndoas.

%\subsection{Beneficiamento primário do cacau}

% Beneficiamento do cacau

%O folder \cite{folder_beneficiamento_ceplac} publicado pela CEPLAC do Pará fala sobre o passo a passo do beneficiamento, com figuras indicando ferramentas a passos do beneficiamento, e algumas tabelas com métricas para a construção de um cocho, e sobre os períodos de estiagem e chuva para que o cacau tenha uma boa fermentação.

%\subsection{Caracterização de amêndoas e chocolate de diferentes variedades de cacau visando a melhoria da qualidade tecnológica}

%O trabalho \cite{silva2013caracterizaccao} aborda sobre a qualidade do cacau, fatores que influenciam no sabor do chocolate, avaliação da qualidade do cacau, como a prova de corte e análise sensorial, o pré-processamento e processamento do cacau, que aborda desde a colheita e abertura dos frutos, até a moagem para obtenção do liquor e a prensagem para obtenção da torta e manteiga de cacau. Aborda também sobre o processamento do chocolate, tratando da mistura e refino, conchgem e temperagem, resfriamento e moldagem.

%A pesquisa trabalha sobre duas safras, atuando na caracterisação fisico-quimica, realizando análise descritiva e quantitativa, e o teste de aceitação dos chocolates.


%\subsection{Caracterização das sementes de variedades de cacau \textit{Theobroma cacao L.} resistentes à vassoura de bruxa durante a fermentação e após a secagem}

%O trabalho \cite{cruz2013caracterizaccao} possui como foco a caracterização das sementes de cacau resistentes à vassora de bruxa.

%No capítulo 1 apresenta o pré-processamento do cacau, que é abordado a colheita e quebra dos frutos, a fermentação, a secagem e o armazenamento. No tópico seguinte, aborda a avaliação da qualidade das amêndoas através da prova de corte. Aborda superficialmente a prova de corte.

%No capítulo 2 é abordado apenas o cacau proveniente de variedades resistentes à vassoura de bruxa.

\section{Conteúdo aqui, conteúdo aqui}
Lorem ipsum dolor sit amet, consectetur adipiscing elit. Vivamus eu magna cursus, mattis velit et, pulvinar lorem. Integer ut nulla eget tellus luctus pellentesque. Nullam fermentum arcu sed tristique congue. Sed ut augue a turpis imperdiet maximus vitae accumsan nisi. Fusce vitae dapibus orci. Pellentesque rutrum tincidunt turpis, id blandit libero lacinia eu. Nunc imperdiet dolor scelerisque ex tristique, nec elementum sem iaculis. Suspendisse eu leo sapien. Vivamus facilisis ultrices sollicitudin. Nunc suscipit, velit eget pretium rhoncus, erat nisl facilisis nulla, quis ultricies orci tellus sed ante. Sed pellentesque, nisl ac venenatis feugiat, augue enim consequat augue, sed porttitor sem purus in enim. Praesent vel tellus eu libero vehicula sodales. Integer tincidunt faucibus sollicitudin. Sed dapibus convallis metus, vel accumsan sem convallis ut.

%\subsection{RGB}
%Amplamente conhecido, é um modelo de cor aditiva onde as cores vermelho, verde e azul são combinadas em diferentes intensidades produzindo outras cores. O RGB (vermelho, verde e azul) são chamadas de cores primárias.

%Na Figura \ref{fig:cor_RGB} podemos ver a representação do espaço de cor RGB.

%\begin{figure}[hbtp!]
% \centering
% \caption{Representação do espaço de cor RGB}
% \includegraphics[scale=0.07]{figs/RGB.png}
% \legend{Fonte: \cite{cor_RGB}}
% \label{fig:cor_RGB}
%\end{figure}

%\subsection{HSV}
%Hue (matiz): define o componente de cor, ou a posição no círculo.

%Saturation (saturação): define o quão "pura" é a cor, ou se ela está
%misturada com outras cores (complementar), tornando-a mais pálida.

%Value (valor/brilho): define a quantidade de luz na mistura, quanto
%mais luz mais clara a cor (na ausência de valor, a imagem é toda
%preta).

%Na Figura \ref{fig:cor_HSV} podemos ver a representação do espaço de cor HSV.

%\begin{figure}[hbtp!]
% \centering
% \caption{Representação do espaço de cor HSV}
% \includegraphics[scale=0.07]{figs/HSV.png}
% \legend{Fonte: \cite{cor_HSV}}
% \label{fig:cor_HSV}
%\end{figure}

\section{Conteúdo aqui, conteúdo aqui}
Lorem ipsum dolor sit amet, consectetur adipiscing elit. Vivamus eu magna cursus, mattis velit et, pulvinar lorem. Integer ut nulla eget tellus luctus pellentesque. Nullam fermentum arcu sed tristique congue. Sed ut augue a turpis imperdiet maximus vitae accumsan nisi. Fusce vitae dapibus orci. Pellentesque rutrum tincidunt turpis, id blandit libero lacinia eu. Nunc imperdiet dolor scelerisque ex tristique, nec elementum sem iaculis. Suspendisse eu leo sapien. Vivamus facilisis ultrices sollicitudin. Nunc suscipit, velit eget pretium rhoncus, erat nisl facilisis nulla, quis ultricies orci tellus sed ante. Sed pellentesque, nisl ac venenatis feugiat, augue enim consequat augue, sed porttitor sem purus in enim. Praesent vel tellus eu libero vehicula sodales. Integer tincidunt faucibus sollicitudin. Sed dapibus convallis metus, vel accumsan sem convallis ut.

%Na próxima seção é apresentado o OpenCV e alguns algoritmos que auxiliaram no processo de segmentação e que foram analisados neste trabalho.

%\newpage
%\subsection{Baseado em detecção de bordas}
%Primeiro aplica-se o método da morfologia matemática para detecção de bordas, que podem ser o de Sobel, Canny, Laplaciana, Prewit ou Roberts. Em seguida é feita um agrupamento de pixels detectados como bordas, a partir de um algoritmo de união ou realce de bordas, que permite determinar de maneira mais precisa o contorno dos objetos de uma imagem.
%Na Figura \ref{Border Detection} pode ser visto a utilização da técnica de detecção de bordas em manchas na pele.

%\begin{figure}[hbpt!]
% \centering
% \caption{Exemplo de segmentação utilizando detecção de bordas}
% \includegraphics[scale=0.3,angle=90]{figs/algoritmos/anisotropic_dermoscopy.png}
% \legend{Fonte: \cite{anisotropic_dermoscopy}}
% \label{Border Detection}
%\end{figure}

%\subsection{Baseado em regiões}
%Um conjunto de pixels que possuem determinado grau de similaridade, são tidos como regiões. No método de segmentação baseado em regiões, cada região é composta por pixels com um valor similar, baseado em um critério de similaridade. Na Figura \ref{Baseado em regiões} pode-se ver a técnica sendo aplicada na segmentação do milho e ervilha, ou de feijões de cores e tamanhos diferentes.

%\begin{figure}[hbpt!]
%\centering
%\caption{Exemplo de segmentação baseado em regiões}
%\begin{minipage}{.5\textwidth}
  %\centering
  %\includegraphics[width=.75\linewidth]{figs/opencv/region_growing_1.png}
%   \label{fig:test1}
%\end{minipage}%
%\begin{minipage}{.5\textwidth}
  %\centering
  %\includegraphics[width=.75\linewidth]{figs/opencv/region_growing_2.pn%g}
%   \label{fig:test2}
%\end{minipage}
%\legend{Fonte: \cite{region_growing}}
%\label{Baseado em regiões}
%\end{figure}


%\subsection{Transformação divisória (Watershed)}

%São algoritmos que possuem como base a morfologia matemática, que permite extrair as bordas existentes em uma imagem, e é também uma técnica de segmentação baseada em regiões. Imagina-se os valores dos pixels da imagem como um gráfico topográfico 3D, onde o 'x' e 'y' são coordenadas do plano e 'z' são os valores dos pixels. O objetivo principal é encontrar as linhas divisórias em uma imagem para separar diferentes regiões, que correspondem aos mínimos do gradiente morfológico. Na Figura \ref{Watershed} pode ser visto a utilização da técnica do watershed para segmentação de uma imagem.

%\begin{figure}[hbpt!]
% \centering
% \caption{Exemplo de segmentação utilizando watershed}
% \includegraphics[scale=0.4]{figs/algoritmos/elephant.jpg}
% \legend{Fonte: \cite{watershed_citation}}
% \label{Watershed}
%\end{figure}

\section{Conteúdo aqui, conteúdo aqui}
\label{sec:opencv}

Lorem ipsum dolor sit amet, consectetur adipiscing elit. Vivamus eu magna cursus, mattis velit et, pulvinar lorem. Integer ut nulla eget tellus luctus pellentesque. Nullam fermentum arcu sed tristique congue. Sed ut augue a turpis imperdiet maximus vitae accumsan nisi. Fusce vitae dapibus orci. Pellentesque rutrum tincidunt turpis, id blandit libero lacinia eu. Nunc imperdiet dolor scelerisque ex tristique, nec elementum sem iaculis. Suspendisse eu leo sapien. Vivamus facilisis ultrices sollicitudin. Nunc suscipit, velit eget pretium rhoncus, erat nisl facilisis nulla, quis ultricies orci tellus sed ante. Sed pellentesque, nisl ac venenatis feugiat, augue enim consequat augue, sed porttitor sem purus in enim. Praesent vel tellus eu libero vehicula sodales. Integer tincidunt faucibus sollicitudin. Sed dapibus convallis metus, vel accumsan sem convallis ut.

%OpenCV-Python é uma API Python para OpenCV, combinando as melhores qualidades da API OpenCV para C++ e a linguagem Python \cite{opencv_python_api}.

%A biblioteca OpenCV-Python é destinada a resolução de problemas de visão computacional, ela faz uso do Numpy, que é uma biblioteca altamente otimizada para operações numéricas que utiliza o estilo de sintaxe do MATLAB. Todas as estruturas de array do OpenCV são convertidas para e de arrays Numpy. Isso permite facilitar a integração com outras bibliotecas que utilizam Numpy como a SciPy e Matplotlib \cite{opencv_python_api}.

%O threshold se baseia na diferença de tons de cinza que compõem diferentes objetos na imagem. A partir das características dos objetos que se quer isolar (obtidos por meio de um histograma por exemplo), a imagem será segmentada em dois grupos: os que possuem níveis de cinza abaixo do valor estabelecido, e os que possuem níveis de cinza acima do valor estabelecido. Para a geração de uma imagem limiarizada, atribui-se um valor fixo para todos os pixels de um mesmo grupo. A imagem gerada será binária, ou seja,  terá  apenas  dois valores  possíveis  para  cada  pixel. Na Figura \ref{Thresholding} podemos ver um exemplo de utilização de thresholding em olhos.

%\begin{figure}[hbpt!]
% \centering
% \caption{Exemplo de utilização do thresholding}
% \includegraphics[angle=270,scale=0.4]{figs/algoritmos/pathologies_thresholding.png}
% \legend{Fonte: \cite{pathologies_thresholding}}
% \label{Thresholding}
%\end{figure}

\subsection{Conteúdo aqui, conteúdo aqui}

Lorem ipsum dolor sit amet, consectetur adipiscing elit. Vivamus eu magna cursus, mattis velit et, pulvinar lorem. Integer ut nulla eget tellus luctus pellentesque. Nullam fermentum arcu sed tristique congue. Sed ut augue a turpis imperdiet maximus vitae accumsan nisi. Fusce vitae dapibus orci. Pellentesque rutrum tincidunt turpis, id blandit libero lacinia eu. Nunc imperdiet dolor scelerisque ex tristique, nec elementum sem iaculis. Suspendisse eu leo sapien. Vivamus facilisis ultrices sollicitudin. Nunc suscipit, velit eget pretium rhoncus, erat nisl facilisis nulla, quis ultricies orci tellus sed ante. Sed pellentesque, nisl ac venenatis feugiat, augue enim consequat augue, sed porttitor sem purus in enim. Praesent vel tellus eu libero vehicula sodales. Integer tincidunt faucibus sollicitudin. Sed dapibus convallis metus, vel accumsan sem convallis ut.

%O OpenCV conta com mais de 150 métodos de conversão do espaço de cores, mas o BGR <-> Cinza e BGR <-> HSV são os mais utilizados. A função tem o escopo Imgproc.cvtColor(imagem\_entrada, imagem\_saida, flag) onde \textit{flag} determina o tipo de conversão.

%Para RGB -> Cinza utiliza-se a \textit{flag} Imgproc.COLOR\_BGR2GRAY e para RGB -> HSV utiliza-se a \textit{flag} cv2.COLOR\_BGR2HSV. Um exemplo de conversão de escala de cor de RGB (Figura \ref{fig:lenna_rgb}) para Cinza (Figura \ref{fig:lenna_gray}) pode ser visualizado na Figura \ref{fig:rgb2gray}.

Lorem ipsum dolor sit amet, consectetur adipiscing elit. Vivamus eu magna cursus, mattis velit et, pulvinar lorem. Integer ut nulla eget tellus luctus pellentesque. Nullam fermentum arcu sed tristique congue. Sed ut augue a turpis imperdiet maximus vitae accumsan nisi. Fusce vitae dapibus orci. Pellentesque rutrum tincidunt turpis, id blandit libero lacinia eu. Nunc imperdiet dolor scelerisque ex tristique, nec elementum sem iaculis. Suspendisse eu leo sapien. Vivamus facilisis ultrices sollicitudin. Nunc suscipit, velit eget pretium rhoncus, erat nisl facilisis nulla, quis ultricies orci tellus sed ante. Sed pellentesque, nisl ac venenatis feugiat, augue enim consequat augue, sed porttitor sem purus in enim. Praesent vel tellus eu libero vehicula sodales. Integer tincidunt faucibus sollicitudin. Sed dapibus convallis metus, vel accumsan sem convallis ut.

\subsection{Conteúdo aqui, conteúdo aqui}
\label{subsec:thresholding}

Lorem ipsum dolor sit amet, consectetur adipiscing elit. Vivamus eu magna cursus, mattis velit et, pulvinar lorem. Integer ut nulla eget tellus luctus pellentesque. Nullam fermentum arcu sed tristique congue. Sed ut augue a turpis imperdiet maximus vitae accumsan nisi. Fusce vitae dapibus orci. Pellentesque rutrum tincidunt turpis, id blandit libero lacinia eu. Nunc imperdiet dolor scelerisque ex tristique, nec elementum sem iaculis. Suspendisse eu leo sapien. Vivamus facilisis ultrices sollicitudin. Nunc suscipit, velit eget pretium rhoncus, erat nisl facilisis nulla, quis ultricies orci tellus sed ante. Sed pellentesque, nisl ac venenatis feugiat, augue enim consequat augue, sed porttitor sem purus in enim. Praesent vel tellus eu libero vehicula sodales. Integer tincidunt faucibus sollicitudin. Sed dapibus convallis metus, vel accumsan sem convallis ut.

%As transformações morfológicas são operações simples baseadas na forma da imagem. Normalmente utilizado em imagens binárias. A função necessita de dois parâmetros, um é a imagem de entrada, o outro é o elemento estruturante, ou kernel, que decide a natureza da operação. Duas operações morfológicas básicas são a erosão e a dilatação.

%A ideia básica da erosão é servir como uma erosão do solo, removendo ruídos ou reduzindo bordas do objeto. A dilatação é a ideia oposta da erosão, cujo objetivo é aumentar as bordas do objeto. Normalmente, em caso de remoção de ruídos, a erosão é sucedida de dilatação.

Lorem ipsum dolor sit amet, consectetur adipiscing elit. Vivamus eu magna cursus, mattis velit et, pulvinar lorem. Integer ut nulla eget tellus luctus pellentesque. Nullam fermentum arcu sed tristique congue. Sed ut augue a turpis imperdiet maximus vitae accumsan nisi. Fusce vitae dapibus orci. Pellentesque rutrum tincidunt turpis, id blandit libero lacinia eu. Nunc imperdiet dolor scelerisque ex tristique, nec elementum sem iaculis. Suspendisse eu leo sapien. Vivamus facilisis ultrices sollicitudin. Nunc suscipit, velit eget pretium rhoncus, erat nisl facilisis nulla, quis ultricies orci tellus sed ante. Sed pellentesque, nisl ac venenatis feugiat, augue enim consequat augue, sed porttitor sem purus in enim. Praesent vel tellus eu libero vehicula sodales. Integer tincidunt faucibus sollicitudin. Sed dapibus convallis metus, vel accumsan sem convallis ut.

\subsection{Operações aritméticas}
Lorem ipsum dolor sit amet, consectetur adipiscing elit. Vivamus eu magna cursus, mattis velit et, pulvinar lorem. Integer ut nulla eget tellus luctus pellentesque. Nullam fermentum arcu sed tristique congue. Sed ut augue a turpis imperdiet maximus vitae accumsan nisi. Fusce vitae dapibus orci. Pellentesque rutrum tincidunt turpis, id blandit libero lacinia eu. Nunc imperdiet dolor scelerisque ex tristique, nec elementum sem iaculis. Suspendisse eu leo sapien. Vivamus facilisis ultrices sollicitudin. Nunc suscipit, velit eget pretium rhoncus, erat nisl facilisis nulla, quis ultricies orci tellus sed ante. Sed pellentesque, nisl ac venenatis feugiat, augue enim consequat augue, sed porttitor sem purus in enim. Praesent vel tellus eu libero vehicula sodales. Integer tincidunt faucibus sollicitudin. Sed dapibus convallis metus, vel accumsan sem convallis ut.

\section{Conteúdo aqui, conteúdo aqui}
\label{sec:java}

Lorem ipsum dolor sit amet, consectetur adipiscing elit. Vivamus eu magna cursus, mattis velit et, pulvinar lorem. Integer ut nulla eget tellus luctus pellentesque. Nullam fermentum arcu sed tristique congue. Sed ut augue a turpis imperdiet maximus vitae accumsan nisi. Fusce vitae dapibus orci. Pellentesque rutrum tincidunt turpis, id blandit libero lacinia eu. Nunc imperdiet dolor scelerisque ex tristique, nec elementum sem iaculis. Suspendisse eu leo sapien. Vivamus facilisis ultrices sollicitudin. Nunc suscipit, velit eget pretium rhoncus, erat nisl facilisis nulla, quis ultricies orci tellus sed ante. Sed pellentesque, nisl ac venenatis feugiat, augue enim consequat augue, sed porttitor sem purus in enim. Praesent vel tellus eu libero vehicula sodales. Integer tincidunt faucibus sollicitudin. Sed dapibus convallis metus, vel accumsan sem convallis ut.

\section{Conteúdo aqui, conteúdo aqui}
\label{sec:android}

Lorem ipsum dolor sit amet, consectetur adipiscing elit. Vivamus eu magna cursus, mattis velit et, pulvinar lorem. Integer ut nulla eget tellus luctus pellentesque. Nullam fermentum arcu sed tristique congue. Sed ut augue a turpis imperdiet maximus vitae accumsan nisi. Fusce vitae dapibus orci. Pellentesque rutrum tincidunt turpis, id blandit libero lacinia eu. Nunc imperdiet dolor scelerisque ex tristique, nec elementum sem iaculis. Suspendisse eu leo sapien. Vivamus facilisis ultrices sollicitudin. Nunc suscipit, velit eget pretium rhoncus, erat nisl facilisis nulla, quis ultricies orci tellus sed ante. Sed pellentesque, nisl ac venenatis feugiat, augue enim consequat augue, sed porttitor sem purus in enim. Praesent vel tellus eu libero vehicula sodales. Integer tincidunt faucibus sollicitudin. Sed dapibus convallis metus, vel accumsan sem convallis ut.

\subsection{Scrum}

Lorem ipsum dolor sit amet, consectetur adipiscing elit. Vivamus eu magna cursus, mattis velit et, pulvinar lorem. Integer ut nulla eget tellus luctus pellentesque. Nullam fermentum arcu sed tristique congue. Sed ut augue a turpis imperdiet maximus vitae accumsan nisi. Fusce vitae dapibus orci. Pellentesque rutrum tincidunt turpis, id blandit libero lacinia eu. Nunc imperdiet dolor scelerisque ex tristique, nec elementum sem iaculis. Suspendisse eu leo sapien. Vivamus facilisis ultrices sollicitudin. Nunc suscipit, velit eget pretium rhoncus, erat nisl facilisis nulla, quis ultricies orci tellus sed ante. Sed pellentesque, nisl ac venenatis feugiat, augue enim consequat augue, sed porttitor sem purus in enim. Praesent vel tellus eu libero vehicula sodales. Integer tincidunt faucibus sollicitudin. Sed dapibus convallis metus, vel accumsan sem convallis ut.

\subsection{Test Driven Development}

Lorem ipsum dolor sit amet, consectetur adipiscing elit. Vivamus eu magna cursus, mattis velit et, pulvinar lorem. Integer ut nulla eget tellus luctus pellentesque. Nullam fermentum arcu sed tristique congue. Sed ut augue a turpis imperdiet maximus vitae accumsan nisi. Fusce vitae dapibus orci. Pellentesque rutrum tincidunt turpis, id blandit libero lacinia eu. Nunc imperdiet dolor scelerisque ex tristique, nec elementum sem iaculis. Suspendisse eu leo sapien. Vivamus facilisis ultrices sollicitudin. Nunc suscipit, velit eget pretium rhoncus, erat nisl facilisis nulla, quis ultricies orci tellus sed ante. Sed pellentesque, nisl ac venenatis feugiat, augue enim consequat augue, sed porttitor sem purus in enim. Praesent vel tellus eu libero vehicula sodales. Integer tincidunt faucibus sollicitudin. Sed dapibus convallis metus, vel accumsan sem convallis ut.
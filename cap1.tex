\chapter{Introdução}

% Contextualizar.....

No ambiente universitário, muitos alunos necessitam algum complemento de renda para manter sua permanência e estudos. Dessa forma, optam por vender artefatos, alimentos, colares, roupas, dentre outros produtos para a comunidade acadêmica, bem como seus arredores.
Visto isso, o atual trabalho propõe um meio tecnológico, simples e objetivo para facilitar o processo de mercantil nas universidades.

\section{O problema}

A principal forma utilizada pelos alunos para divulgar a venda de produtos no ambiente acadêmico é por meio de redes sociais como o Whatsapp, Instagram e Facebook. Porém essas ferramentas descentralizadas, em muitos casos, não alcançam o efeito desejado sobre  o trâmite de compra e venda.

\section{Objetivos}

\subsection{ Objetivo Geral}

Solucionar o problema apresentado por meio de recursos tecnológicos. Será implementado um sistema web para centralizar esse processo entre o vendedor e comprador no ambiente acadêmico.

\subsection{Objetivos Específicos}
\begin{enumerate}
    \item Elaborar uma análise de requisitos para o problema.
    \item Desenvolver um sistema web que solucione o problema.
    \item Aplicar uma metodologia ágil durante o desenvolvimento.
    \item Disponibilizar o sistema para a comunidade acadêmica.
    \item Analisar o impacto do sistema no ambiente acadêmico.
\end{enumerate}

\section{Justificativa}
\label{sec:justificativa}

Através da realização deste projeto, espera-se ampliar a circulação financeira entre os membros da comunidade acadêmica, facilitando, assim, o comércio informal e, consequentemente, melhorando a autonomia financeira do estudante.


\section{Organização}
Lorem ipsum dolor sit amet, consectetur adipiscing elit. Vivamus eu magna cursus, mattis velit et, pulvinar lorem. Integer ut nulla eget tellus luctus pellentesque. Nullam fermentum arcu sed tristique congue. Sed ut augue a turpis imperdiet maximus vitae accumsan nisi. Fusce vitae dapibus orci. Pellentesque rutrum tincidunt turpis, id blandit libero lacinia eu. Nunc imperdiet dolor scelerisque ex tristique, nec elementum sem iaculis. Suspendisse eu leo sapien. Vivamus facilisis ultrices sollicitudin. Nunc suscipit, velit eget pretium rhoncus, erat nisl facilisis nulla, quis ultricies orci tellus sed ante. Sed pellentesque, nisl ac venenatis feugiat, augue enim consequat augue, sed porttitor sem purus in enim. Praesent vel tellus eu libero vehicula sodales. Integer tincidunt faucibus sollicitudin. Sed dapibus convallis metus, vel accumsan sem convallis ut.
    
% Introduzir Figura
%\begin{figure}
%	   \centering
%	   		\includegraphics[scale=0.35]{figs/MPCbase.PNG} 
%	   \caption{Algoritmo MPC}
%	   \label{label para referencia cruzada Figura}
%\end{figure}


% Lista de Item

%\begin{itemize}
%	\item item 1
%	\item item 2
%	\item item 3
%\end{itemize}

% Equação
%\begin{equation}
%	\label{label para referencia cruzada %equacoes} 
%	y(t)=\sum_{i=1}^{\infty}h_i\Delta u(t-i)
%\end{equation}

% Equação em linha 
%$\hat{y}(t+k\mid t)= \sum^\infty_{i=1} g_i %\Delta u(t+k-i\mid t)$

% Citação -  Criei o arquivo de bibliografia usando o jabref

%\cite{Camacho2007} 

% Referencia Cruzada de Figura
%\ref{label para referencia cruzada Figura}

% Referencia Cruzada de Equação
%\ref{label para referencia cruzada equacoes}


% Tabelas

%\begin{table}[h]
%\begin{center}
%     \caption{Índices 1 para casos factíveis}
%     \begin{tabular}{| l | l | l | l |}
%     \hline Índice & LP Petro & LP 2 & %Diferença\\ 
%     \hline $SES_y$& 60.5406 & 60.5492 & %-0.0087\\
%     \hline $SES_u$& 1166.1464 & 1166.1464 & 2.36*$10^{-9}$ \\
%     \hline
 
%    \end{tabular}
%\label{table:indices1}
%\end{center}
%\end{table}
\chapter{Introdução}

% Contextualizar.....

No ambiente universitário, muitos alunos necessitam algum complemento de renda para manter sua permanência e estudos. Dessa forma, optam por vender artefatos, alimentos, colares, roupas, dentre outros produtos para a comunidade acadêmica, bem como seus arredores.

O advento de novas tecnologias da informação consolidou um novo paradigma econômico, sendo possível a criação de novos mercados, o que contribuiu para a venda de produtos e serviços por meio dessa tecnologia. Diante desse novo modelo de mercância, espera-se que o âmbito universitário tire proveito, buscando a aproximar pessoas interessadas a vender de pessoas interessadas a comprar produtos e serviços.

Visto isso, o atual trabalho propõe um meio tecnológico, simples e objetivo para facilitar o processo mercantil informal nas universidades do Brasil.

\section{O problema}

A principal forma utilizada pelos vendedores para divulgação de produtos no ambiente acadêmico é por meio de redes sociais como o Whatsapp, Instagram e Facebook. Porém essas ferramentas descentralizadas, em muitos casos, não alcançam o efeito desejado sobre  o trâmite de compra e venda.

Tal descentralização é ruim para o vendedor, sendo necessário manipular e administrar diversas ferramentas para um mesmo objetivo. Do mesmo modo se apresenta ruim para os clientes, aos quais necessitam ter o número e Instagram de diversos vendedores para realizar a comunicação necessária para realizar as compras. 

Mas, e se esses dados estivessem reunidos em um só lugar? E se fosse possível o acesso ao preço dos produtos de cada vendedor, bem como seus dados de contato? E se fosse possível realizar a devida compra com poucas ações? Esse é o grande problema da descentralização da informação mercantil nas universidades.

\section{Justificativa}
\label{sec:justificativa}

Como será melhor abordado na seção \ref{trabalho_informal}, o principal motivo para evasão universitária são as dificuldades financeiras que os alunos passam. O segundo motivo é a dificuldade de conciliar trabalho e estudo. Diante desses motivos, se faz necessário a elaboração de uma solução que contribua para que a condição financeira do aluno seja melhorada, passando assim a conseguir se manter na universidade e possivelmente não necessitar de trabalhos externos a universidade para que seja possível se manter na mesma.

Através da realização deste projeto, espera-se ampliar a circulação financeira entre os membros da comunidade acadêmica, facilitando, assim, o comércio informal e, consequentemente, melhorando a autonomia financeira do estudante.

\section{Objetivos}

\subsection{ Objetivo Geral}

Solucionar o problema apresentado por meio de recursos tecnológicos. Será implementado um sistema web para centralizar esse processo entre o vendedor e comprador no ambiente acadêmico.

\subsection{Objetivos Específicos}
\begin{enumerate}
    % \item Elaborar uma análise de requisitos para o problema.
    \item Desenvolver um sistema web que solucione o problema
    % \item Aplicar uma metodologia ágil durante o desenvolvimento.
    \item Validar e confiabilizar o sistema por meio de testes automatizados.
    \item Disponibilizar o código fonte da aplicação
    \item Comparar o sistema final com o mockup proposto
    \item Disponibilizar o sistema para a comunidade acadêmica
    \item Analisar o impacto do sistema no ambiente acadêmico. (????)
\end{enumerate}


\section{Organização}
O presente trabalho está dividido em 5 capítulos, e está organizado da seguinte forma:

No Capítulo II encontra-se o referencial teórico, apresentando todos os fundamentos teóricos que solidificam e criam uma base para o entendimento do trabalho como um todo.

No Capítulo III está explicado toda a metodologia que foi utilizada para chegar ao resultado, mostrando cada passo de forma detalhada.

Os resultados alcançados serão abordados no Capítulo IV, ao qual será realizado uma comparação com os objetivos que foram propostos.

No Capítulo V será abordado uma discussão acerca do resultado obtido, bem como apresentar possíveis os trabalhos futuros necessários para uma boa continuação do trabalho.
    
% Introduzir Figura
%\begin{figure}
%	   \centering
%	   		\includegraphics[scale=0.35]{figs/MPCbase.PNG} 
%	   \caption{Algoritmo MPC}
%	   \label{label para referencia cruzada Figura}
%\end{figure}


% Lista de Item

%\begin{itemize}
%	\item item 1
%	\item item 2
%	\item item 3
%\end{itemize}

% Equação
%\begin{equation}
%	\label{label para referencia cruzada %equacoes} 
%	y(t)=\sum_{i=1}^{\infty}h_i\Delta u(t-i)
%\end{equation}

% Equação em linha 
%$\hat{y}(t+k\mid t)= \sum^\infty_{i=1} g_i %\Delta u(t+k-i\mid t)$

% Citação -  Criei o arquivo de bibliografia usando o jabref

%\cite{Camacho2007} 

% Referencia Cruzada de Figura
%\ref{label para referencia cruzada Figura}

% Referencia Cruzada de Equação
%\ref{label para referencia cruzada equacoes}


% Tabelas

%\begin{table}[h]
%\begin{center}
%     \caption{Índices 1 para casos factíveis}
%     \begin{tabular}{| l | l | l | l |}
%     \hline Índice & LP Petro & LP 2 & %Diferença\\ 
%     \hline $SES_y$& 60.5406 & 60.5492 & %-0.0087\\
%     \hline $SES_u$& 1166.1464 & 1166.1464 & 2.36*$10^{-9}$ \\
%     \hline
 
%    \end{tabular}
%\label{table:indices1}
%\end{center}
%\end{table}
\chapter{Introdução}

% Contextualizar.....

De acordo com os dados de 2018 da OIT\cite{grimshaw2019resumen}, cerca de 61\% dos trabalhadores do mundo atuam de maneira informal, isso equivale a cerca de 2 bilhões de pessoas. Já no Brasil, essa informalidade chega a atingir 41,3\% dos trabalhadores em 2019, segundo dados do IBGE. Do mesmo modo, no ambiente universitário muitos alunos necessitam de alguma renda para manter sua permanência e estudos\cite{ribeiro2005evasao}. Dessa forma, optam por vender artefatos, alimentos, colares, roupas, dentre outros produtos para a comunidade acadêmica, bem como seus arredores.

O advento de novas tecnologias da informação consolidou um novo paradigma econômico, sendo possível a criação de novos mercados, o que contribuiu para a venda de produtos e serviços por meio dessa tecnologia. Diante desse novo modelo de mercancia, espera-se que o âmbito universitário tire proveito, buscando a aproximar pessoas interessadas a vender de pessoas interessadas a comprar produtos e serviços.

Visto isso, o atual trabalho propõe um meio tecnológico, simples e objetivo para facilitar o processo mercantil informal na universidade.

\section{O problema}

A principal forma utilizada pelos vendedores para divulgação de produtos no ambiente acadêmico é por meio de redes sociais como o Whatsapp, Instagram e Facebook. Porém essas ferramentas descentralizadas, em muitos casos, não alcançam o efeito desejado sobre  o trâmite de compra e venda.

Tal descentralização é ruim para o vendedor, sendo necessário manipular e administrar diversas ferramentas para um mesmo objetivo. Do mesmo modo se apresenta ruim para os clientes, aos quais necessitam ter o número e Instagram de diversos vendedores para realizar a comunicação necessária para realizar as compras.

Mas, e se esses dados estivessem reunidos em um só lugar? E se fosse possível o acesso ao preço dos produtos de cada vendedor, bem como seus dados de contato? E se fosse possível realizar a devida compra com poucas ações? Esse é o grande problema da descentralização da informação mercantil nas universidades.

A centralização desses produtos em um único sistema facilitará ao cliente encontrar e ter acesso a grande diversidade de produtos e vendedores que existem na universidade. Ao ter acesso a toda essas informações reunidas dos produtos, o cliente conseguirá fazer uma melhor escolha em qual produto comprar, já que conseguirá comparar facilmente os preços e antigas avaliações realizadas por outros clientes.

Ao centralizar os produtos na plataforma, o vendedor poderá ter um controle melhor sobre o quê e quanto vendeu em relação aos seus produtos. Outro ponto positivo seria uma melhor visibilidade dos seus produtos, visto que eles serão apresentados de forma pública e gratuita em um sistema que tem a finalidade específica para um público específico ao qual o próprio vendedor está incluído.

\section{Justificativa}
\label{sec:justificativa}

Como será melhor abordado na seção \ref{trabalho_informal}, de acordo com \cite{ribeiro2005evasao}, o principal motivo para evasão de estudantes de uma universidade são as dificuldades financeiras que os alunos passam, o segundo motivo é a dificuldade de conciliar trabalho e estudo. Diante desses motivos, se faz necessário a elaboração de uma solução que contribua para que a condição financeira do aluno seja melhorada, passando assim a conseguir se manter na universidade e possivelmente não necessitar de trabalhos externos a universidade para que seja possível se manter na mesma.

Através da realização deste projeto, espera-se facilitar o comércio informal e, consequentemente, melhorar a autonomia financeira do estudante servindo assim como subsídio para o estudante se manter na universidade.

\section{Objetivos}

\subsection{Objetivo Geral}
Propor uma aplicação gratuita de \textit{e-commerce} C2C(\textit{Consumer to Consumer}), voltada para o comércio informal na Universidade Estadual de Santa Cruz.

\subsection{Objetivos Específicos}
O atual projeto tem alguns objetivos específicos, são eles:
\begin{enumerate}
    \item Criar mecanismos para o gerenciamento e compartilhamento de produtos à venda;
    \item Possibilitar a listagem de produtos por filtros;
    \item Permitir aos compradores fazerem pedidos de produtos;
    \item Desenvolver interfaces para avaliação dos usuários;
    \item Validar o sistema por meio de testes automatizados;
    \item Disponibilizar o código fonte do sistema.
\end{enumerate}


\section{Organização}
O presente projeto está dividido em 5 capítulos, e organizado da seguinte forma:

No Capítulo II encontra-se o referencial teórico, apresentando todos os fundamentos teóricos que solidificam e criam uma base para o entendimento do trabalho como um todo.

No Capítulo III está explicado toda a metodologia que foi utilizada para chegar ao resultado, mostrando cada passo de forma detalhada.

Os resultados alcançados e discussões serão abordados no Capítulo IV, ao qual será realizado uma comparação com os objetivos que foram propostos.

No Capítulo V será discutida a conclusão acerca do trabalho, bem como apresentar possíveis os trabalhos futuros necessários para uma boa continuação do projeto.
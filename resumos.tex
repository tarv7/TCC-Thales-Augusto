% ---
% RESUMOS
% ---

% resumo em português
\setlength{\absparsep}{18pt} % ajusta o espaçamento dos parágrafos do resumo
\begin{resumo}
Foi identificado que muitas pessoas no mundo inteiro atuam de maneira informal, no ambiente universitários isso não é diferente, muitos discentes necessitam buscar formas de complementar sua renda para se manter presente na universidade. Dessa forma optam por vender artefatos, alimentos, colares, roupas, dentre outros produtos para a comunidade acadêmica, bem como seus arredores. Mas a forma de divulgação dos produtos pelos vendedores não é efetiva devido aos meios utilizados, meios esses que são as redes sociais, isso torna a informação descentralizada. O atual projeto tem como objetivo principal facilitar e organizar a venda dos produtos vendidos pelos estudantes por meio de um sistema web. Para desenvolvimento desta aplicação foi utilizada uma metodologia que buscasse por meio da elaboração de diagramas e da técnica de desenvolvimento \textit{Test Driven Development}(TDD) buscar validar as implementações do sistema. Por meio desse projeto, espera-se facilitar o comércio informal e, consequentemente, melhorar a autonomia financeira do estudante para que o mesmo consiga se manter. Como resultado foi implementado o sistema Web e disponibilizado para comunidade acadêmica, bem como seu código fonte. Foram implementadas funcionalidades de listagem de produtos, listagem de vendedores, módulo de pedidos, dentre outras funcionalidades. Por fim, concluímos que conseguimos atingir os parcialmente os objetivos propostos para o sistema e será preciso melhorar o trabalho em alguns pontos, como: expandir o sistema para outras universidade, incluir métodos de pagamento online, dentre outros.


 \textbf{Palavras-chave}: web, mercância, universidade.
\end{resumo}

% resumo em inglês
% \begin{resumo}[Abstract]
%  \begin{otherlanguage*}{english}
 % Realizar a verificação ortográfica, sintaxe e semântica.

%    \vspace{\onelineskip}
 
%    \noindent 
%    \textbf{Keywords}: .
%  \end{otherlanguage*}
% \end{resumo}

%% resumo em francês 
%\begin{resumo}[Résumé]
% \begin{otherlanguage*}{french}
%    Il s'agit d'un résumé en français.
% 
%   \textbf{Mots-clés}: latex. abntex. publication de textes.
% \end{otherlanguage*}
%\end{resumo}
%
%% resumo em espanhol
%\begin{resumo}[Resumen]
% \begin{otherlanguage*}{spanish}
%   Este es el resumen en español.
%  
%   \textbf{Palabras clave}: latex. abntex. publicación de textos.
% \end{otherlanguage*}
%\end{resumo}
% ---

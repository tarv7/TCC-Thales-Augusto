% ---
% RESUMOS
% ---

% resumo em português
\setlength{\absparsep}{18pt} % ajusta o espaçamento dos parágrafos do resumo
\begin{resumo}
Foi identificado que muitos discentes necessitam buscar formas de complementar sua renda para se manter presente na universidade. Dessa forma optam por vender artefatos, alimentos, colares, roupas, dentre outros produtos para a comunidade acadêmica, bem como seus arredores. Mas a forma de divulgação dos produtos pelos vendedores não é efetiva devido aos meios utilizados. O atual projeto tem como objetivo principal centralizar e organizar a venda dos produtos vendidos pelos estudantes por meio de um sistema web. Para desenvolvimento desta aplicação foi utilizada a metodologia Test Driven Development(TDD), buscando assim validar as implementações do sistema. Por meio desse projeto, espera-se ampliar a circulação financeira entre os membros da comunidade acadêmica, facilitando, assim, o comércio informal e, consequentemente, melhorando a autonomia financeira do estudante para que o mesmo consiga se manter. Os resultados alcançados serão analisados por meio de análises de comparação aos objetivos propostos, mostrando assim se a abrangência e impacto do sistema no meio acadêmico foi alcançado como esperado.


 \textbf{Palavras-chave}: web, mercância, universidade.
\end{resumo}

% resumo em inglês
% \begin{resumo}[Abstract]
%  \begin{otherlanguage*}{english}
 % Realizar a verificação ortográfica, sintaxe e semântica.

%    \vspace{\onelineskip}
 
%    \noindent 
%    \textbf{Keywords}: .
%  \end{otherlanguage*}
% \end{resumo}

%% resumo em francês 
%\begin{resumo}[Résumé]
% \begin{otherlanguage*}{french}
%    Il s'agit d'un résumé en français.
% 
%   \textbf{Mots-clés}: latex. abntex. publication de textes.
% \end{otherlanguage*}
%\end{resumo}
%
%% resumo em espanhol
%\begin{resumo}[Resumen]
% \begin{otherlanguage*}{spanish}
%   Este es el resumen en español.
%  
%   \textbf{Palabras clave}: latex. abntex. publicación de textos.
% \end{otherlanguage*}
%\end{resumo}
% ---

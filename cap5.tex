\chapter{Conclusão e trabalhos futuros}
\label{chap:conclusoes_trabalhos_futuros}

Este trabalho demostrou a modelagem e a elaboração de um sistema web que facilita o processo de mercancia entre as pessoas do âmbito universitário, atualmente o sistema já encontra-se em uso pela comunidade acadêmica em torno da UESC. Por meio dele é possível realizar compras/vendas de produtos, comunicação direta com o vendedor/comprador e avaliar o vendedor.

Desse modo, conclui-se a partir dos resultados obtidos, apresentados na seção \ref{chap:resultados}, que os objetivos propostos foram alcançados com êxito. Porém, por meio de \textit{feedbacks} de usuários reais do sistema e observações da equipe executante do trabalho foram identificados alguns pontos que precisam ser melhorados em trabalhos futuros, são eles:
\begin{itemize}
    \item Estudo do impacto do sistema no âmbito universitário
    \item Expansão do sistema para outras universidades
    \item Criação de uma área administrativa que seja possível cadastrar novas universidades e categorias no sistema
    \item Tornar pública a página individual dos vendedores, para que seja possível visualizar seus dados mesmo sem estar logado
    \item Melhorar a responsividade da aplicação
 \end{itemize}

% \section{Méritos e Limitações}

% \subsection{Contribuições pessoais}
% O que o trabalho contribuiu para o seu aprendizado pessoal?